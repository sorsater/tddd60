\documentclass[12pt,a4paper]{article}
\usepackage[utf8]{inputenc}
\usepackage{fancyhdr}
\usepackage{datetime2}

\pagestyle{fancy}
\fancyhf{}

\lhead{micso554}
\rhead{\today} % yyyy-mm-dd
\setlength{\headheight}{15pt}

\cfoot{\thepage}

\begin{document}

\begin{center}
    \Huge
    \textbf{Interaktionsdesign och UX}

    \vspace{0.3cm}
    \Large
    Michael Sörsäter
    
    \vspace{0.7cm}
    \textbf{Sammanfattning}
\end{center}

\section{Konceptfasen}
Designkoncept är ett viktigt och väldigt användbart verktyg för att bearbeta idéer kring en produkt. Genom att använda sig av flertalet storyboards för att exemplifiera möjliga användningar arbetas ett koncept fram. 

När ett designkoncept arbetas fram är det viktigt att prioritera och ha klart för sig vilka som är målgruppen och vilken betydelse produkten kommer ha för användaren.

Det finns ett flertal arbetssätt för att ta fram designkoncept. Brainstorming är känt och används för att på kort tid ta fram många (och gärna tokiga) förslag. Syftet med många förslag är att få ett brett perspektiv på idéen samt försöka få gruppen att tänka i nya banor. En liknande, men mer strukturerad, metod är 365 som även kallas för brainwriting. Metoden bygger på iterationer där de olika gruppmedlemmarna tar fram tre radikala idéer. Vid efterföljande iterationer kan gruppen fortsätta bearbeta och vidareutveckla idéerna från tidigare iterationer och förfina dem. Det finns flera arbetssätt liknande brainstorming och 365 för att ta fram idéer men de utgår ofta från någon av dessa.

Efter att gruppen tagit fram ett antal olika designkoncept ska gruppen bestämma sig för vilket av dem de ska arbeta med. Ofta är alternativ liknande varandra alternativt kompletterande och man använder båda, en så kallad syntes av koncepten. För att ta fram det bästa konceptet används en Pugh-matris. Utifrån denna fattar gruppen beslut om vilket koncept de ska arbeta vidare med.

\section{Bearbetningsfasen}
Efter att konceptet tagits fram i konceptfasen ska konceptet omarbetas och testas hur det fungerar i verkliga livet. Frågor som hur ska det fungera, hur ska gränssnittet se ut är drivande i denna fas. 

För att kunna arbeta med konceptet bryts det ner i sina beståndsdelar och ritas upp i uppgiftsflödesdiagram. Från detta diagram tas \textbf{objekt} och \textbf{handlingar} ut för att underlätta skrivandet av kravspecifikationen.

Utifrån konceptets beståndsdelar och kravspecifikation kommer nästa steg som är framtagandet av ett gränssnitt. Stegen för gränssnitt är många. Den initiala skissen lägger grunden för hela produkten och den bör göras ordentlig. De olika komponenterna i gränssnittet ska bestämmas och tas fram och de ska alla få tilldelade funktioner.

I gränssnittsdesignen är det viktigt att tänka på användarens upplevelse och användande av produkten. Det är även i denna fas som en eventuell grafisk profil tas fram för att standardisera designen i produkten. 

För att få återkoppling på gränssnittet görs tester på olika användargrupper för att 

\subsection{Bearbetningsfasens insikter}
\subsection{Bearbetningsfasens avsikter}
\subsection{Bearbetningsfasens idéer}
\subsection{Bearbetningsfasens värderingar}

\section{Detaljeringsfasen}

\iffalse

\section{Koncepfasens idéer och värderingar}
\subsection{Konceptfasens idéer}
\subsection{Konceptfasens värderingar}

\section{Bearbetningsfasen}
\subsection{Bearbetningsfasens insikter}
\subsection{Bearbetningsfasens avsikter}
\subsection{Bearbetningsfasens idéer}
\subsection{Bearbetningsfasens värderingar}

\section{Detaljeringsfasen, överlämning och avslutning}
\subsection{Detaljeringsfasens insikter}
\subsection{Detaljeringsfasens avsikter}
\subsection{Detaljeringsfasens idéer}
\subsection{Detaljeringsfasens värderingar}
\subsection{Överlämning}
\subsection{Uppföljning och reflektion}
\subsection{Avslutande ord}

\fi
\end{document}